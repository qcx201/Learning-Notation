\documentclass{article}[12pt]
\usepackage[utf8]{inputenc}
\usepackage[a4paper, margin=1in]{geometry}
\usepackage[english]{babel}
\usepackage{amssymb, amsmath, amsthm} % math symbols
\usepackage{hyperref} % hyperlinks
\usepackage{multicol}

% title
\title{
    Learning Notation: Seminar One Exercises \\
    Logic and Proof Techniques
    \vspace{-2em}
    }
\author{}
\date{March 7, 2022}

% paragraph /indent spacing
\setlength{\parskip}{6pt}
\setlength{\parindent}{0pt}

% mathtools equation numbering
\counterwithin*{equation}{section}
\renewcommand\theequation{\thesection.\arabic{equation}}

% amsthm theorems formatting
\newtheorem{theorem}{Theorem}

\newtheorem{conjecture}{Conjecture}
\newtheorem*{conjecture*}{Conjecture}

\newtheorem{proposition}{Proposition}[section]
\newtheorem*{proposition*}{Proposition} % unnumbered

\newtheorem{definition}{Definition}[section]
\newtheorem*{definition*}{Definition} % unnumbered

\newtheorem{exercise}{Exercise}[section]
\newtheorem*{exercise*}{Exercise} % unnumbered


% special symobls
\newcommand{\N}{\mathbb{N}}
\newcommand{\Z}{\mathbb{Z}}
\newcommand{\Q}{\mathbb{Q}}
\newcommand{\R}{\mathbb{R}}
\newcommand{\C}{\mathbb{C}}
\newcommand{\PS}{\mathcal{P}}

% text box for definitions and theorems
\newcommand{\textbox}[1]{\fbox{\parbox{\textwidth}{#1}}}


\begin{document}

    \maketitle
    
    \renewcommand{\labelenumi}{(\alph{enumi})}
    
    
    \setcounter{section}{1}
    \section{Predicate Logic}
            
    \subsection{Conjunction and disjunction}
    
        \begin{exercise}
            Verify these algebraic properties with truth tables.
            \begin{align}
                \text{Commutativity:}\quad&&
                    x \land y
                    \equiv
                    y \land x,
                    \quad&
                    x \lor y
                    \equiv
                    y \lor x.
                \\\text{Associativity:}\quad&&
                    x \land (y \land z)
                    \equiv
                    (x \land y) \land z,
                    \quad&
                    x \lor (y \lor z)
                    \equiv
                    (x \lor y) \lor z.
                \\\text{Distributivity:}\quad&&
                    x \land (y \lor z)
                    \equiv
                    (x \land y) \lor (x \land z),
                    \quad&
                    x \lor (y \land z)
                    \equiv
                    (x \lor y) \land (x \lor z).
                \\\text{Identities:}\quad&&
                    x \land T \equiv x,
                    \quad&
                    x \lor F \equiv x.
                \\\text{Annihilators:}\quad&&
                    x \land F \equiv F,
                    \quad&
                    x \lor T \equiv T.
            \end{align}
        \end{exercise}
        
    \subsection{De Morgan's laws}
    
        \begin{exercise}
            Prove the second de Morgan's law, that $P \lor Q  \equiv \neg (\neg P \land \neg Q)$.
        \end{exercise}
    
    \subsection{Quantifiers}
        
        \begin{exercise}
            The set of natural numbers is $\R$. Consider the following statements:
            \begin{align}
                \text{There exists a real number $a$ where for any real number $x$, $ax = x$.}
                \\
                \text{There exists a real number $b$ where for a any real number $x$, $bx = b$.}
            \end{align}
            Can you rewrite these statements with quantifiers? Are these statements true? If so, what are $a$ and $b$?
        \end{exercise}
        
        \begin{exercise}
            Consider the following statement:
            \begin{align}
                \lim_{x \to p} f(x) = L
                \iff
                \bigg(
                \forall \epsilon > 0, \ 
                \exists \delta > 0, \
                | x - p | < \delta \implies | f(x) - L | < \epsilon
                \bigg)
                .
            \end{align}
            This is called the \textbf{epsilon-delta definition} of limit. Can you restate the definition in words? Can you interpret what it means?
        \end{exercise}
        
    
    \subsection{Conditional and biconditional statements}
    
        \begin{exercise}
            Show that $P \implies Q$ is logically equivalent to $\neg P \lor (P \land Q)$.
        \end{exercise}
        \begin{exercise}
            Express $P \iff Q$ as negations, conjunctions, and disjunctions of $P$ and $Q$.
        \end{exercise}
        
        
        \newpage
        
        \begin{exercise}
            Which of the following statements are true?
            \begin{multicols}{2}
                \begin{enumerate}
                    \item
                    $x < 3 \implies x \le 4$
                    
                    \item
                    $x < 3 \impliedby x \le 4$
                    
                    \item
                    $x > y \implies x \ge y$
                    
                    \item
                    $x > y \impliedby x \ge y$
                    
                    \item
                    $[(P \implies Q) \land (\neg P)] \implies  (Q \equiv T)$
                    
                    \item
                    $[(P \implies Q) \land (\neg P)] \implies  Q$
                    
                    \item
                    $[(P \implies Q) \land P] \implies Q$
                    
                    % \item
                    % $(|x| < y) \land (y > 0) \iff (x > -y) \land (x < y)$
                    
                    % \item
                    % $(|x| < y) \land (y > 0) \iff (x > -y) \lor (x < y)$
                    
                    % \item
                    % $(|x| > y) \land (y > 0) \iff (x > y) \land (x < -y)$
                    
                    % \item
                    % $(|x| > y) \land (y > 0) \iff (x > y) \lor (x < -y)$
                \end{enumerate}
            \end{multicols}
        \end{exercise}
    
    
    \subsection{Tautology}
    
        \begin{exercise}
            Is each of the following a tautology, contradiction, or neither?
            \begin{multicols}{2}
                \begin{enumerate}
                    \item
                    $(x < 0) \lor (x > 0)$
                    
                    \item
                    $(x < 0) \land (x > 0)$
                    
                    \item
                    $(x < y) \land (x \ge y)$
                    
                    \item
                    $[(P \implies Q) \land (\neg Q)] \implies  \neg P$
                    
                    \item
                    $[(P \lor Q) \land (\neg Q)] \implies  \neg P$
                    
                    \item
                    $(P \land Q) \iff Q$
                    
                    \item
                    $Q \iff (P \lor Q)$
                    
                \end{enumerate}
            \end{multicols}
            
        \end{exercise}
        
        \textbox{
        \textbf{Solution \theexercise.}
        \begin{multicols}{2}
                \begin{enumerate}
                    \item
                    $(x < 0) \lor (x > 0)$. Neither, since $x = 0$ is also possible.
                    
                    \item
                    $(x < 0) \land (x > 0)$. Contradiction, since $x$ cannot be greater and less than 0.
                    
                    \item
                    $(x < y) \land (x \ge y)$. Tautology.
                    
                    \item
                    $[(P \implies Q) \land (\neg Q)] \implies  \neg P$. Tautology.
                    
                    \item
                    $[(P \lor Q) \land (\neg Q)] \implies  \neg P$.  Contradiction, since $[(P \lor Q) \land (\neg Q)] \implies  P$.
                    
                    \item
                    $(P \land Q) \iff Q$. Neither, since the conditional is true but not the converse.
                    
                    \item
                    $Q \iff (P \lor Q)$. Neither, since the conditional is also true but not the converse.
                    
                \end{enumerate}
            \end{multicols}
            }
        
        \begin{exercise}
            Verify that the following statements are tautologies.
            
            \begin{multicols}{2}
            \begin{enumerate}
                \item
                $P \iff \neg (\neg P)$
                % Double negation
                
                \item 
                $P \lor Q  \iff \neg (\neg P \land \neg Q)$
                % De Morgan's law
                
                \item
                $P \land Q \iff \neg (\neg P \lor \neg Q)$
                % De Morgan's law
                
                \item\label{eqn:contraposition}
                $(P \implies Q) \iff (\neg Q \implies \neg P)$
                % Contraposition
                
                \item \label{eqn:contradiction}
                $P \iff (\neg P \implies Q) \land (\neg P \implies \neg Q)$
                % Reductio ad absurdum
                
                \item
                $((A \implies B) \land (B \implies C)) \implies (A \implies C)$
                % Syllogism
                
                \item\label{eqn:pf-cases}
                $[(A \lor B) \land (A \implies C) \lor (B \implies C)] \implies C$
                % proof by cases

            \end{enumerate}
            \end{multicols}
            
        \end{exercise}
        
    % \newpage
    \section{Proof Techniques}
        
        
    \subsection{Direct proof}
    
        \textbox{
        \begin{definition}[Odd integer]
            An integer $x$ is odd $\iff$ there exists an integer $y$ where $x = 2y + 1$.
        \end{definition}
        
        \begin{definition}[Even integer]
            An integer $x$ is even $\iff$ there exists an integer $y$ where $x = 2y$.
        \end{definition}
        }
    
        \begin{exercise}\label{prop:even-prod}
            Prove that if $x$ is an even integer, then $x y$ is even for any integer $y$.
        \end{exercise}
        
        \begin{exercise}
            Prove that if $x$ is an odd integer, then $x^3$ is odd.
        \end{exercise}
        
    \subsection{Proof by cases}
    
        \begin{exercise}
            Prove that $|x| \ge y \implies (x \le -y) \lor (x \ge y)$.
        \end{exercise}
        
        \begin{exercise}\label{prop:even-sum}
            Prove that if $x, y$ are either both even or both odd, then $x + y$ is even.
        \end{exercise}
        
        \begin{exercise}\label{prop:odd-sum}
            Prove that if $x, y$ are not both even or both odd, then $x + y$ is odd.
        \end{exercise}
        

    \subsection{Proof by contrapositive}
    
        \textbox{
        \begin{definition}[Informal]
            The derivative of a function $f(x)$ at $x$ can be approximated as
            \begin{align}
                f'(x)
                \simeq \frac{f(x + dx) - f(x)}{dx}
                \simeq \frac{f(x) - f(x - dx)}{dx},
                \ dx > 0.
            \end{align}
            Assume these approximations hold as equalities for an (infinitesimally) small $dx > 0$.
        \end{definition}
        }
        
        \begin{exercise}
            Prove that for a differentiable function $f(x)$, if $f(x^*)$ is the minimum, then $f'(x^*) = 0$.
        \end{exercise}
        
        \begin{exercise}
            Prove that for any real numbers $x$ and $y$, if $y^3 + yx^2 \le x^3 + xy^2$, then $y \le x$.
        \end{exercise}
        
    \subsection{Proof by contradiction}
        
        \textbox{
        \begin{definition}[Rationality]
            A number $q$ is \textbf{rational} if it can be expressed as a quotient of two integers without common denominators.
            \begin{align}
                q \in \Q
                \iff
                \exists\ m, n \in \Z,\ 
                q = \frac{m}{n}.
            \end{align}
            
            If $q \in \R$ and $q$ is not rational, then it is \textbf{irrational}.
        \end{definition}
        }
        
        \begin{exercise}
            Prove that there are infinitely many natural numbers, supposing that the following statements are true about $\N$, the set of all natural numbers.
            \begin{enumerate}
                \item Zero is a natural number.
                
                \item $\forall n \in \N$, $n + 1$ exists and is a natural number.
                
                \item $\forall n \in \N$, $n+1 \ne 0$.
                
                \item $\forall n, m \in N$, if $n \ne m$, then $n + 1 \ne m + 1$.
            \end{enumerate}
        \end{exercise}
        
        \begin{exercise}
            Prove that there is no smallest positive rational number.
        \end{exercise}
        
        \begin{exercise}
            Prove that $\sqrt{3}$ is irrational.
        \end{exercise}
        
        % \begin{exercise}
        %     Prove that $\sqrt[3]{2}$ is irrational.
        % \end{exercise}
        
        \begin{exercise}
            For $n \in \N,\ n > 1$, show that if $p$ is prime, then $\sqrt[n]{p}$ is irrational.
        \end{exercise}
        
    \subsection{Proof by induction}
    
        
        \textbox{
        \begin{definition}[Fibonacci]
            The Fibonacci sequence $\{F_n\}_1^\infty$ follows $F_1 = 1,\ F_2 = 2,$ and for $n > 2$,
            \begin{align}
                F_{n} = F_{n-2} + F_{n-1}
            \end{align}
        \end{definition}
        }

        
        \begin{exercise}
            Show by induction that for any $n \ge 1$ and any $x \in \R$,
            \begin{align}
                \sum_{m=0}^n x = n x.
            \end{align}
        \end{exercise}
        
        \begin{exercise}
            Show that for any $r \ne 1$,
            \begin{align}
                \sum_{m=0}^n r^m = \frac{1 - r^{n+1}}{1 - r}.
            \end{align}
        \end{exercise}
        
        \begin{exercise}
            Show that for any $n \ge 1$ and any $x \in \R$,
            \begin{align}
                \sum_{m=0}^n m^2 = \frac{n(n-1)(2n-1)}{6}.
            \end{align}
        \end{exercise}
        
        % \begin{exercise}
        %     Show that
        %     \begin{align}
        %         \exists x \in X, P(x)
        %         &\implies
        %         \neg (\forall x \in X, \neg P(x)),
        %         \\
        %         \forall x \in X, P(x)
        %         &\implies
        %         \neg (\exists x \in X, \neg P(x)).
        %     \end{align}
        % \end{exercise}
        
        \begin{exercise}
            Prove the binomial theorem, that for any natural $n \ge 0$,
            \begin{align}
                (x + y)^n
                &= \sum_{k=0}^n \binom{n}{k} x^{n-k} y^k,\quad
                \text{where}\quad
                \binom{n}{k} = \frac{n!}{k! (n-k)!}.
            \end{align}
        \end{exercise}
        
        
\end{document}