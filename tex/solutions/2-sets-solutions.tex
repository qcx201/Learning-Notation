\documentclass{article}[12pt]
\usepackage[a4paper, margin=1in]{geometry}
\usepackage[utf8]{inputenc}
\usepackage[english]{babel}
\usepackage{amssymb, amsmath, amsthm} % math symbols
\usepackage{csquotes} % format quote blocks
\usepackage{hyperref} % hyperlinks
\usepackage[table]{xcolor} % tables
\usepackage{tikz} % to draw Venn diagrams

\hypersetup{
    colorlinks=true,
    linkcolor=black,
    citecolor=black,
    urlcolor=blue,
    pdfborderstyle={/S/U/W 1}
    }

% title
\title{Learning Notation: Seminar Two \\ Set Theory}
\author{Jack (Quan Cheng) Xie}
\date{\today}

% paragraph /indent spacing
\setlength{\parskip}{6pt}
\setlength{\parindent}{0pt}


% mathtools equation numbering
\counterwithin*{equation}{section}
\renewcommand\theequation{\thesection.\arabic{equation}}

% amsthm theorems formatting
\newtheorem{theorem}{Theorem}

\newtheorem{conjecture}{Conjecture}
\newtheorem*{conjecture*}{Conjecture} % unnumbered

\newtheorem{proposition}{Proposition}
\newtheorem*{proposition*}{Proposition} 

\newtheorem{claim}{Claim}
\newtheorem*{claim*}{Claim}

\newtheorem{definition}{Definition}
\newtheorem*{definition*}{Definition} % unnumbered

% question counter
\newcounter{question}
\newenvironment{question}[1][]{\refstepcounter{question}\par\noindent\textbf{Question~\thequestion. #1} \rmfamily}{}

\newcommand\questionnum{\refstepcounter{question}\thequestion}


% special symobls
\newcommand{\N}{\mathbb{N}}
\newcommand{\Z}{\mathbb{Z}}
\newcommand{\Q}{\mathbb{Q}}
\newcommand{\R}{\mathbb{R}}
\newcommand{\C}{\mathbb{C}}
\newcommand{\PS}{\mathcal{P}}

% text box for definitions and theorems
\newcommand{\textbox}[1]{\fbox{\parbox{\textwidth}{#1}}}


\begin{document}

    \section{Questions}
    
    
    \subsection*{Question \questionnum}
        
        For a set $A$, does $a_1, a_2, a_3 \in A$ imply $A = \{a_1, a_2, a_3\}?$
        
        \textbox{
        No. For example, $A = \{a_1, a_2, a_3, a_4\}$.
        }
    
    
    \subsection*{Question \questionnum}
        
        What are $|D|$, $|\varnothing|$, and $|\{\varnothing\}|$?
        \begin{align*}
            A &= \{a, b, c\}, \quad
            B = \{1, 2, 3\},
            \\
            C &= \{\varnothing, A, B\}, \quad
            D = \{\varnothing, a, b, c, 1, 2, 3, C\}.
        \end{align*}
    
    
    \subsection*{Question \questionnum}
        Show that $\sqrt{3}$ is irrational.
        
        \textbox{
            \begin{proposition*}
                $\sqrt{3}$ is irrational.
            \end{proposition*}
        
        \begin{proof}
            By contradiction, suppose $\sqrt{3}$ is rational. Then there exists $m, n \in \N$ without common factors where
            \begin{align}
                \sqrt{3} = \frac{m}{n}.
            \end{align}
            Then we have that
            \begin{align}
                3 = \frac{m^2}{n^2} \implies m^2 = 3 n^2 \implies m = 3p, \ p = \frac{n^2}{3}
            \end{align}
            which means 3 is a factor of $m$. However,
            \begin{align}
                p = \frac{n^2}{3} \implies n^2 = 3p \implies n = 3q,
            \end{align}
            which implies 3 is also a factor of $n$. Then 3 cannot be both a factor of $m$ and $n$ if they have no common factors.
            
        \end{proof}
        }
    
    \subsection*{Question \questionnum}
    
        Construct $\Z$ and $\Q$ from $\N$ with set builder notation.
        
        \textbox{
        \begin{align}
            \Z &= \{n : n \in \N\} \cup \{-n : n \in \N\},
            \\
            \Q &= \left\{\frac{m}{n}: m, n \in \N \right\} \cup \left\{-\frac{m}{n}: m, n \in \N \right\}
        \end{align}
        }
    
    
    \subsection*{Question \questionnum}
    
        What is $\PS(\varnothing)$? What is $\PS(\PS(\varnothing))$ and $\PS(\PS(\PS(\varnothing)))$?
        
        \textbox{
        \begin{align}
            \PS(\varnothing) &= \{\varnothing\},
            \\
            \PS(\PS(\varnothing)) &= \{\varnothing, \{\varnothing\}\}
            \\
            \PS(\PS(\PS(\varnothing))) &= \{\varnothing, \{\varnothing\}, \{\{\varnothing\}\}, \{\varnothing, \{\varnothing\}\}\}
        \end{align}
        }
    
    
    \subsection*{Question \questionnum}
    
        Prove by induction that $|\PS(S)| = 2^{|S|}$ for any finite set $S$.
        
        \textbox{
            \begin{proposition*}
                $|\PS(S)| = 2^{|S|}$ for any finite set $S$.
            \end{proposition*}
            
            \begin{proof}
                Let $\{S_n\}_{n=0}^\infty$ be a sequence of sets where $|S_n| = n$ and $S_{n} \subset S_{n+1}$. Denote the elements of $S_n = \{e_1, ..., e_n\}$.
                
                \textbf{Base case ($n=0$):} We can show that $|\PS(S_0)| = 2^0 = 1$. By definition of the empty set that it contains no members, $S_0 = \varnothing$. Then
                    \begin{align}
                        n = 1 \implies S_n = S_0 = \varnothing \implies \PS(S_n) = \{\varnothing\} \implies |\PS(S_n)| = 1.
                    \end{align}
                    
                \textbf{Inductive step:} We want to show that $|\PS(S_n)| = 2^{n}  \implies |\PS(S_{n+1})| = 2^{n + 1}$.
                \begin{align}
                    \PS(S_{n+1})
                    &= \PS(S_n \; \cup \; \{e_{n+1}\})
                    \\
                    &= \PS(S_n) \; \dot\cup \; \{ A : A \subseteq S_{n+1} \land \{e_{n+1}\} \subseteq A \}
                    \\
                    &= \PS(S_n) \; \dot\cup \; \{ A \; \dot\cup \;\{e_{n+1}\} : A \subseteq \PS(S_n) \}
                \end{align}
                
                Note that $| \{ A \; \dot\cup \;\{e_{n+1}\} : A \subseteq \PS(S_n) \} | = |\PS(S_n)| = 2^n$ by the induction hypothesis. Then
                \begin{align}
                    | \PS(S_{n+1}) |
                    &= | \PS(S_n) | + | \{ A \; \dot\cup \;\{n+1\} : A \subseteq \PS(S_n) \} | \\
                    &= 2^n + 2^n = 2^{n+1},
                \end{align}
                which completes the induction proof.
            \end{proof}
        }
    
    
    \subsection*{Question \questionnum}
        
        Show that the set of naturals $\N$ and the set of even numbers are equicardinal.
        
        \textbox{Let $f(n) = 2n$. Then this is a bijection from $\N$ to the even numbers, which means the naturals are equicardinal to the set of even numbers (even though the even numbers is a proper subset!).}
    
    \subsection*{Question \questionnum}
    
        Show that the cardinality of $[0, 1]$ is equal to that of any interval $[a, b]$.
        
        \textbox{
            For any $a, b \in \R$, $f(x) = a + (b-a)x$ is a bijection from $[0, 1] \to [a, b]$.
        }
        
    \subsection*{Question \questionnum}
    
        Show that $\R$ and $\R^+$ are equicardinal.
        
        \textbox{
            The monotonic function $f(x) = \ln x$ is a bijection from $\R^{+} \to \R$.
        }
    
    \subsection*{Question \questionnum}
    
        See appendix \eqref{app:interval-cardinalities} for proofs that $[0, 1]$, $[0, 1)$, and $(0, 1)$ are equicardinal. Show that $[0, 1]$ is equicardinal to $\R$.
    
        \textbox{
            We showed that $|[0, 1]| = |(0, 1)|$ in appendix \eqref{app:interval-cardinalities}, and we showed that $|\R| = |\R^+|$ in the previous question. Since $f(x) = -\ln x$ is a bijection from $(0, 1) \to \R^{+}$, they are also equicardinal. Then we have that $|[0, 1]| = |(0, 1)| = |\R^+| = |\R|$.
        }
    
    \appendix
    \section{Appendix}
        
        \subsection{Equicardinality of intervals}\label{app:interval-cardinalities}
        
        \textbox{
            \begin{proposition*}
                The intervals $(0, 1)$ , $[0, 1)$, $(0, 1]$, and $[0, 1]$ are equicardinal.
            \end{proposition*}
        }
        
        \begin{proof}
            We can show there exist bijective functions from $(0, 1)$ to $[0, 1), (0, 1],$ and $[0, 1]$ by Hilbert's Hotel. Let $H = \left\{\frac{1}{n} : n \in \N, n > 1 \right\}$.
            
            Define $f_1 : (0, 1) \to [0, 1)$ as
            \begin{align}
                f_1(x) = \begin{cases}
                    0,              & x = \frac{1}{2}
                    \\
                    \frac{1}{n-1},  & x \in \left\{\frac{1}{n} : n \in \N, n > 2\right\}
                    \\
                    x,              & x \not\in H.
                \end{cases}
            \end{align}
            
            Define $f_2 : (0, 1) \to (0, 1]$ as
            \begin{align}
                f_2(x) = \begin{cases}
                    1,              & x = \frac{1}{2}
                    \\
                    \frac{1}{n-1},  & x \in \left\{\frac{1}{n} : n \in \N, n > 2\right\}
                    \\
                    x,              & x \not\in H.
                \end{cases}
            \end{align}
            
            Define $f_3 : (0, 1) \to [0, 1]$ as
            \begin{align}
                f_3(x) = \begin{cases}
                    0,              & x = \frac{1}{2}
                    \\
                    1,              & x = \frac{1}{3}
                    \\
                    \frac{1}{n-2},  & x \in \left\{\frac{1}{n} : n \in \N, n > 3\right\}
                    \\
                    x,              & x \not\in H.
                \end{cases}
            \end{align}
            
        Each of these functions is a bijection. Then the open interval is equicardinal to each of the other type of interval, which means they are all equicardinal.
        \end{proof}
        
\end{document}